\documentclass{beamer}
\usepackage[utf8]{inputenc}
\usepackage{multirow}
\usepackage{colortbl}
\usepackage{tikz}
\usepackage{amsmath}
% \usepackage{anyfontsize}
% \usepackage{lmodern}
%\usepackage{svg}

\usetheme[
titlepagelogo=img/logo_unimib.pdf,
color=red,
language=italian,
bullet=triangle,
coding=utf8,
pageofpages=di,
assistantsupervisor=true
]{TorinoTh}
\author{Giacomini Stefano}
\rel{Prof. Gianluca Della Vedova}
\title{Simulated Annealing per l'inferenza di mutazioni\\ ricorrenti in alberi tumorali}
\ateneo{Università degli Studi di Milano-Bicocca}
\date{21 Marzo 2022}
\assistantsupervisor{Dott. Simone Ciccolella}

% \usepackage[dvipsnames]{xcolor}

% \definecolor{darkgreen}{RGB}{20, 155, 82}
% \definecolor{moredarkgreen}{RGB}{14, 84, 46}
\usepackage[beamer,customcolors]{hf-tikz}
\hfsetfillcolor{alerted text.fg!10}
\hfsetbordercolor{alerted text.fg}
\begin{document}
\titlepageframe

\begin{tframe}{Introduzione e scaletta}
  \begin{block}<1->{Introduzione e motivazioni:}
    \begin{itemize}
      \item terapia mirata per la cura di tumori
      \item importanza dell'omoplasia nei dati virali
    \end{itemize}
  \end{block}
  \begin{block}<2->{{Definizioni e implementazione:}}
    \begin{itemize}
      \item Problemi di riscostruzione di alberi filogenetici
      \item Dollo-k e Camin-Sokal-k
      \item Simulated Annealing
      \item Implementazione delle funzionalità
    \end{itemize}
  \end{block}
\end{tframe}

\begin{tframe}{Definizioni: Problema di riscostruzione di alberi filogenetici}
    \vspace{5mm}
    \begin{itemize}
      \item \textit{matrice di input}
      \item \textit{probabilità $\alpha,\ \beta,\ \gamma\ e\ \delta$}
      \item \textit{profilo del genotipo}
    \end{itemize}

    \vspace{10mm}
    \tiny
    \begin{equation*}
      \tikzmarkin<1->{a}(6.7,-0.95)(-0.2,0.45)
        \max{\sum_{j}^{m}[-c_{j}\log(1-P(L(j)))-f_{j}\log(1-P(D(j)))+\sum_{i}^{n}\log(P(I_{ij}|D(T, \sigma_{i})_{j}))]}
      \tikzmarkend{a}
    \end{equation*}
\end{tframe}

\begin{tframe}{Definizioni: Dollo-k e Camin-Sokal-k}
  \begin{columns}
  \column{0.35\textwidth}
  \begin{itemize}
    \item \textit{Dollo}
    \item \textit{Camin-Sokal}

    \vspace{5mm}
    \begin{equation*}
      \tikzmarkin<1->{a}
        Car(k, r)
      \tikzmarkend{a}
    \end{equation*}
  \end{itemize}

  \column{0.55\textwidth}
  \includegraphics[scale = 0.23]{img/tree.pdf}
\end{columns}
\end{tframe}

\begin{tframe}{Definizioni: Simulated Annealing}
  \begin{block}<1->{\ }
    \begin{itemize}
      \item funzionamento dell'algoritmo
      \item metodo di riduzione della temperatura
      \item mosse
    \end{itemize}
  \end{block}
\end{tframe}

\begin{tframe}{Implementazione delle funzionalità}
  \begin{block}<1->{\ }
    \begin{itemize}
      \item likelihood
      \item mosse del Simulated Annealing
      \item funzione di controllo
      \item analisi dei tempi
    \end{itemize}
  \end{block}
\end{tframe}


\title{Grazie per l'attenzione}
\ateneo{Ringraziamenti}
\titlepageframe

\end{document}
